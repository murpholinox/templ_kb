% Load the kaobook class
\documentclass[
	fontsize=10pt, % Base font size
	twoside=true, % Use different layouts for even and odd pages (in particular, if twoside=true, the margin column will be always on the outside)
	%open=any, % If twoside=true, uncomment this to force new chapters to start on any page, not only on right (odd) pages
	secnumdepth=1, % How deep to number headings. Defaults to 1 (sections)
	numbers=noenddot, % Comment to output dots after chapter numbers; the most common values for this option are: enddot, noenddot and auto (see the KOMAScript documentation for an in-depth explanation)
]{kaobook}

% Choose the language
\usepackage[spanish,es-nosectiondot,mexico]{babel} % Load characters and hyphenation, the option -es-nosectiondot removes dots after the sections
\usepackage[autostyle=true]{csquotes}

%\usepackage{showframe} % Uncomment to show boxes around the text area, margin, header and footer
%\usepackage{showlabels} % Uncomment to output the content of \label commands to the document where they are used

% Load the bibliography package
\usepackage{kaobiblio}
\addbibresource{bib.bib} % Bibliography file

% Load mathematical packages for theorems and related environments
\usepackage{kaotheorems}

% Load the package for hyperreferences
\usepackage{kaorefs}

% Carga el paquete para escribir unidades del SIU
\usepackage{siunitx}[=v2] % La versión tres no admite Angstroms!!!
\DeclareSIUnit\Molar{\textsc{m}}

% Carga el paquete para escribir reacciones químicas
\usepackage[version=4]{mhchem}

\graphicspath{./images} % Paths where images are looked for

\makeindex[columns=3, title=Alphabetical Index, intoc] % Make LaTeX produce the files required to compile the index


\begin{document}

%----------------------------------------------------------------------------------------
%	BOOK INFORMATION
%----------------------------------------------------------------------------------------

%\titlehead{Document Template}

\title[Efecto del pH en la condición de cristalización sobre el daño por radiación en cristales de proteína]{Efecto del pH en la condición de cristalización sobre el daño por radiación en cristales de proteína}
% \subtitle{Subtitle}

\author[FMP]{Francisco Murphy Pérez}
\date{\today}

%\publishers{An Awesome Publisher}

%----------------------------------------------------------------------------------------

\frontmatter % Denotes the start of the pre-document content, uses roman numerals

%----------------------------------------------------------------------------------------
%	COPYRIGHT PAGE
%----------------------------------------------------------------------------------------

\makeatletter
%\uppertitleback{\@titlehead} % Header

\lowertitleback{
	%	\textbf{Disclaimer} \\
	%	You can edit this page to suit your needs. For instance, here we have a no copyright statement, a colophon and some other information. This page is based on the corresponding page of Ken Arroyo Ohori's thesis, with minimal changes.
	%	
	\medskip
	%	
	%	\textbf{No copyright} \\
	%	\cczero\ This book is released into the public domain using the CC0 code. To the extent possible under law, I waive all copyright and related or neighbouring rights to this work.
	%	
	%	To view a copy of the CC0 code, visit: \\\url{http://creativecommons.org/publicdomain/zero/1.0/}
	%	
	\medskip
	
	\textbf{Colofón} \\
	Este documento se compuso con la ayuda de: \KOMAScript{} (\url{https://sourceforge.net/projects/koma-script/}) y \LaTeX{} (\url{https://www.latex-project.org/}), usando la plantilla denominada kaobook (\url{https://github.com/fmarotta/kaobook/}). 
	
	\medskip
	
	%\textbf{Publisher} \\
	%Primera versión mayo 2019%\@publishers
}
\makeatother

%----------------------------------------------------------------------------------------
%	DEDICATION
%----------------------------------------------------------------------------------------


\dedication{
	A Lucía, por supuesto.	
	
}

%----------------------------------------------------------------------------------------
%	OUTPUT TITLE PAGE AND PREVIOUS
%----------------------------------------------------------------------------------------

% Note that \maketitle outputs the pages before here
\maketitle

%----------------------------------------------------------------------------------------
%	PREFACE
%----------------------------------------------------------------------------------------

\chapter*{Resumen}
Aquí va el resumen, normalmente esto se escribe al final de la tesis.



%----------------------------------------------------------------------------------------
%	TABLE OF CONTENTS & LIST OF FIGURES/TABLES
%----------------------------------------------------------------------------------------

\begingroup % Local scope for the following commands

% Define the style for the TOC, LOF, and LOT
%\setstretch{1} % Uncomment to modify line spacing in the ToC
%\hypersetup{linkcolor=blue} % Uncomment to set the colour of links in the ToC
\setlength{\textheight}{230\vscale} % Manually adjust the height of the ToC pages

% Turn on compatibility mode for the etoc package
\etocstandarddisplaystyle % "toc display" as if etoc was not loaded
\etocstandardlines % "toc lines as if etoc was not loaded

\tableofcontents % Output the table of contents

\listoffigures % Output the list of figures

% Comment both of the following lines to have the LOF and the LOT on different pages
\let\cleardoublepage\bigskip
\let\clearpage\bigskip

\listoftables % Output the list of tables

%\newacronym{api}{API}{Application Programming Interface }
\newacronym{pdb}{PDB}{Protein Data Bank}
%\newacronym{drx}{DRX}{Difracción de rayos X}
%\newacronym{ccd}{CCD}{Charge-coupled device}
%\newacronym{xfel}{XFEL}{X-ray Free Electron Laser}
\newacronym{iupac}{IUPAC}{International Union of Pure and Applied Chemistry}
\setglossarystyle{listgroup} % Set the style of the glossary (see https://en.wikibooks.org/wiki/LaTeX/Glossary for a reference)
\printglossary[title=Acrónimos, toctitle=Acrónimos] % Output the glossary, 'title' is the chapter heading for the glossary, toctitle is the table of contents heading


\endgroup

%----------------------------------------------------------------------------------------
%	MAIN BODY
%----------------------------------------------------------------------------------------

\mainmatter % Denotes the start of the main document content, resets page numbering and uses arabic numbers
\setchapterstyle{kao} % Choose the default chapter heading style

\input{./chaps/introduccion}
\input{./chaps/antecedentes}
\chapter{Hipótesis}

La alta concentración de protones, es decir, bajos niveles de pH en la condición de cristalización, permitirán mitigar el daño por radiación en cristales de proteína. En particular, (\emph{i}), se espera que este efecto de protección se dé a una dosis de radiación absorbida baja, esto por el simple hecho de que la producción de electrones solvatados está dada por el tiempo de exposición al haz de rayos X; en cambio, la cantidad de protones dados en una cierta condición cristalina será finita. Además, (\emph{ii}), se espera que el efecto de protección sea notable en residuos susceptibles al daño por radiación específico provocado por los \ce{e^{-}_{solv.}}.
	

\chapter{Objetivo}
\labch{objet}

El objetivo de este proyecto es determinar el efecto del pH en la condición de cristalización de ciertas proteínas sobre el daño por radiación. 
\section{Objetivos particulares}
\labsec{objpar}
Para lograr el objetivo principal de este proyecto, se plantean los siguientes objetivos particulares:

%Obtener mapas de diferencia de densidad electrónica entre puntos fijos de dosis de radiación absorbida.

\begin{enumerate}
	\item Realizar un análisis \emph{in silico} para determinar qué proteínas son capaces de cristalizar en un intervalo de pH amplio\sidenote{No se espera que el esto sea más de cinco o seis unidades de pH, dada la naturaleza de las proteínas}.
	\item Cristalizar las proteínas seleccionadas a diferentes niveles de pH. 
	\item Determinar los parámetros de la colecta de datos que produzcan niveles similares o idénticos de dosis de radiación absorbida en las diferentes proteínas cristalizadas.
	\item Realizar series de colectas de datos idénticas y continuas hasta alcanzar el límite de Garman.
	\item Procesar los patrones de difracción y afinar un modelo estructural inicial para serie de colectas de datos.
	\item Mapear la diferencia de densidad electrónica entre colectas de datos al modelo inicial de cada proteína y realizar un análisis comparativo, para determinar las diferencias en daño por radiación a diferentes niveles de pH.
\end{enumerate}	
\chapter{Materiales y métodos}
\labch{mayme}

El \acrshort{pdb} contiene información acerca de cientos de miles de proteínas cristalizadas en diferentes condiciones de cristalización. Para hallar qué proteínas pueden cristalizar en un amplio intervalo de pH se realizó un análisis \emph{in silico}, como se explica brevemente\sidenote{Con detalle en \url{https://github.com/murpholinox/doctorado}.} a continuación. 

\subsection{Extracción de datos}
%\sidenote{Actualmente la API fue reemplazada por una nueva versión cuyo desarrollo no ha sido completado \url{https://tinyurl.com/y84kuvs7}.}
Se decidió emplear la información cruda del \acrshort{pdb}, es decir, extraer la información experimental necesaria directamente del cabezal de los archivos de las estructuras depositadas en el \acrshort{pdb}. La principal ventaja es que la extracción de información es de una manera más directa, sin depender de la interfaz de programación de aplicaciones (\acrshort{api}, por sus siglas en inglés) del mismo \acrshort{pdb}. Para extraer la información de los cabezales se usó el programa \Package{gemmi}\sidenote{Disponible en el siguiente enlace \url{https://github.com/project-gemmi/gemmi}.}. La información extraída es la siguiente: el contador de la entidad macromolecular, el tipo de entidad, el código de acceso de la base de datos de referencia\sidenote{La mayoría de las veces el código usado es aquél de la base de datos UniProt \url{https://www.uniprot.org/}.}, su descripción, el método experimental para crecer los cristales, el pH de la condición de cristalización, los detalles del experimento de la cristalización, la resolución final del modelo estructural, el grupo espacial en el que cristaliza la macromolécula y el identificador de objeto digital de la publicación científica correspondiente. A continuación se muestra un ejemplo de la información extraída:

\begin{kaobox}[frametitle=Ejemplo 1]
\begin{verbatim}
6LU7,1,polypeptide(L),P0DTD1,main protease,EVAPORATION,\
6,"2% polyethylene glycol (PEG) 6000, 3% DMSO, 1mM DTT,\
0.1M MES buffer (pH 6.0), protein concentration 5mg/ml,\
VAPOR DIFFUSION, HANGING DROP, temperature 293K",2.16,\
2.16,C 1 2 1,10.1038/s41586-020-2223-y,21728,210031
6LU7,2,polypeptide(L),P0DTD1,main protease,EVAPORATION,\
6,"2% polyethylene glycol (PEG) 6000, 3% DMSO, 1mM DTT,\
0.1M MES buffer (pH 6.0), protein concentration 5mg/ml,\
VAPOR DIFFUSION, HANGING DROP, temperature 293K",2.16,\
2.16,C 1 2 1,10.1038/s41586-020-2223-y,21728,210031
\end{verbatim}
\end{kaobox}

El resultado final es una tabla de datos, con \num{226523} observaciones, o filas, y \num{14} variables, o columnas. El número de observaciones es mayor que el número de estructuras en el \acrshort{pdb}, esto es debido a que cada archivo puede tener más de una macromolécula\sidenote{Como es el caso del ejemplo 1.}. %La integridad de los datos se demuestra con una gráfica.

\subsection{Limpieza de datos}
Se aplican los siguientes filtros a los datos extraídos para eliminar observaciones que:

\begin{enumerate}
	\item Carecen de código de acceso.
	\item Tienen una resolución peor que \SI{2}{\angstrom}. 
	\item Carecen del valor de pH de la condición de cristalización.
	\item El número de entidades es mayor o igual a dos.
\end{enumerate}

La lógica de los filtros es la siguiente: (\num{1} y \num{3}) Remover entradas que no tengan las anotaciones correspondientes\sidenote{Desafortunadamente la información experimental no siempre se encuentra disponible en los archivos del PDB.}, en este caso el código de acceso de la proteína y el pH de la condición de cristalización. La última es la anotación más relevante para este proyecto y la primera es la única manera de conocer casi inequívocamente la proteína representada en el archivo. (\num{2}) La resolución final del modelo estructural es un indicador de la calidad del cristal obtenido, a mayor resolución menos defectos posee el cristal. Este filtro garantizará que el listado de proteínas obtenidas sean fáciles de cristalizar. Además este filtro servirá para el análisis posterior, al comparar el daño por radiación específico, pues los resultados serán más confiables con una buena resolución. (\num{4}) Como se mencionó anteriormente, un archivo puede contener múltiples macromoléculas. Este filtro ayuda a descartar proteínas cristalizadas con otras. En general, las condiciones de cristalización para combinaciones diferentes de macromoléculas serán diferentes, por lo que no tiene caso tener varias observaciones de la misma proteína si presenta diferentes compañeros. 

En la tabla \reftab{tab:top50}, se muestran las 50 proteínas más representadas en los datos que cumplen los primeros cuatros filtros. A partir de las cuales se aplican los siguientes dos filtros, que ayudan a eliminar las observaciones donde:

\begin{enumerate}
	\setcounter{enumi}{4}
	\item Su secuencia de aminoácidos sea diferente de la secuencia consenso para cada conjunto de proteínas.
	\item No presenten un intervalo de pH amplio en su cristalización.
\end{enumerate}

La lógica de estos dos filtros es la siguiente: (\num{5}) en el primer filtro se alegó que el código de acceso de UniProt es la manera de conocer \emph{casi inequívocamente la proteína representada}. En el caso de algunos virus, el código corresponde a un gen que puede codificar para diferentes proteínas. Debido a esto se realizó un alineamiento múltiple, con el programa \Package{mafft} \sidecite{Katoh2013},  para eliminar proteínas distintas a la respectiva secuencia consenso con el mismo código de acceso. A continuación se muestra un ejemplo:

\begin{kaobox}[frametitle=Ejemplo 2]
	\begin{verbatim}
	6W4H,1,polypeptide(L),P0DTD1,SARS-CoV-2 NSP16,...
	6W4H,2,polypeptide(L),P0DTD1,SARS-CoV-2 NSP16,...
	6W6Y,1,polypeptide(L),P0DTD1,SARS-CoV-2 NSP3,...
	6WQD,1,polypeptide(L),P0DTD1,SARS-CoV-2 NSP7,...
	6WQD,2,polypeptide(L),P0DTD1,SARS-CoV-2 NSP7,...
	7BUY,1,polypeptide(L),7BUY,SARS-CoV-2 virus Main protease,
	\end{verbatim}
\end{kaobox}

Es claro que el código, \texttt{P0DTD1}, es el mismo; sin embargo, las observaciones corresponden a diferentes proteínas. Este filtro ayuda a mantener proteínas en las que su secuencia no difiera entre sí por más de 15 aminoácidos. Se mantienen entonces proteínas que difieran entre sí por la etiqueta de polihistidinas, pero se excluyen proteínas que contengan el péptido señal y proteínas quimeras, por ejemplo. Y (\num{6}) es la condición experimental que nos interesa en este proyecto, proteínas que cristalicen en un amplio intervalo de pH. Este filtro se aplicó por partes. Primero se realizó un gráfico de caja para cada una de las 50 proteínas más representadas en los datos. Este tipo de gráfica da una representación visual de la distribución de la variable en cuestión. Si la distribución no cubre al menos tres unidades de pH, entonces se descarta dicha proteína, en caso contrario se mantiene. De las proteínas restantes, 25, se realiza un histograma de frecuencias para determinar de manera cuantitativa la frecuencia con la que cada proteína ha sido cristalizada en valores de pH diferentes. Si la frecuencia no es mayor a cinco para la mayoría de las barras en cada histograma, la proteína se descarta, si no se mantiene (véase la \reffig{fig:vis-anal}). Esto resulta en un listado de 14 proteínas, donde las diez primeras entradas se presentan a continuación (véase la \reftab{tab:list}). 

\begin{table}[h]
	\centering
	\begin{tabular}{@{}llll@{}}
		\toprule
		Número & Código & Intervalo       & Nombre                              \\ \midrule
		1      & P00918 & 6               & Anhidrasa carbónica                 \\
		2      & P00698 & 5.5             & Lisozima                            \\
		3      & P00760 & 6               & Tripsina                            \\
		4      & P02766 & 5               & Transtiretina                       \\
		5      & P42212 & 6               & Proteína verde fluorescente         \\
		6      & O60885 & 3.5             & Proteína bromodominio 4             \\
		7      & P19491 & 4.5             & Receptor de glutamato 2             \\
		8      & O26232 & 4               & Orotidina-5’-fosfato descarboxilasa \\
		9      & P00772 & 4.5             & Elastasa                            \\
		10     & P00644 & 3               & Termonucleasa                       \\
\bottomrule
	\end{tabular}
	\caption[Proteínas que cristalizan en un intervalo amplio de pH]{Proteínas que cristalizan en un intervalo amplio de pH.}
	\labtab{tab:list}
\end{table}

\section{Colecta y análisis de datos}
Los cristales obtenidos serán difractados en un sincrotrón, midiendo la dosis de radiación absorbida. Para cada proteína se tendrán que realizar colectas de datos secuenciales de manera repetitiva. Gracias a la presencia de modelos estructurales en el \acrshort{pdb}, se puede estimar la dosis absorbida por la proteína antes de realizar el experimento de difracción. Esto se puede realizar gracias al programa \Package{raddose} \sidecite{Zeldin2013}. Se resolveran las estructuras por reemplazo molecular y se crearán mapas de diferencia de densidad electrónica para analizar la diferencia en daño por radiación específico a distintos pHs al mismo nivel de dosis de radiación absorbida. %Cabe señalar que la mayor parte del análisis posterior a la colecta de datos será \emph{in silico} y gracias al trabajo anterior se dispone de un flujo de trabajo semiautomático lo que facilitará en gran medida el presente proyecto.
\chapter{Resultados}
	\begin{enumerate}
		\item Del análisis \emph{in silico} para determinar qué proteínas son capaces de cristalizar en un intervalo de pH amplio, se obtuvieron cinco proteínas con los siguientes identificadores de \verb|uniprot| \cite{Bateman2021}: \verb|P42212|, \verb|P00698|, \verb|P00760|, \verb|P00918|, \verb|P02766|. 
		\item  Con respecto a la cristalización de las proteínas seleccionadas$\ldots$ 
		\begin{itemize}
		\item Se han puesto condiciones de cristalización con la lisozima (\verb|P00698|) y la tripsina (\verb|P00760|). Ambas conseguidas de manera comercial y sin procedimientos posteriores de purificación. 
		
		\item De los sistemas de amortiguadores de capacidad extendida desarrollados por Newman \cite{Newman2004}, se utilizó la primera condición de cristalización (denominada de ahora en adelante, C1): ácido succínico, glicina y sodio dihidrógeno fosfato monohidratado a una concentración de \SI{250}{\milli\Molar}. 
		
		\item Se realizaron experimentos de cristalización con la técnica modificada de \emph{microbatch}, propuesta por D'Arcy \emph{et al.} \cite{DArcy1996}. En estos experimentos, se varió la concentración de la proteína (15 o \SI{30}{\milli\gram\per\milli\liter}); la proporción entre aceites de parafina y silicona (1:0, 1:0.5, 1:1, 1:2 y 1:4); el pH (10, 9.5, 9.0, $\ldots$, 4.5) y finalmente, también se cambió el porcentaje de cloruro de sodio (0, 3 y \SI{6}{\percent} p/v) en la condición de cristalización. Esto debido a que, en ciertas condiciones, se ha observado una dependencia del tamaño de los cristales de lisozima obtenidos con respecto a la concentración de cloruro de sodio \cite{Svanidze2005}. Se usaron placas de cristalización de Terasaki Greiner de \num{72} pozos. Esto permite tres corridas de pH, con tres porcentajes de cloruro de sodio por duplicado. Cabe destacar que en cada placa la proporción de aceites es diferente. En total, se tienen \num{720} experimentos (\num{1440}, tomando en cuenta los duplicados). Esto último, tiene que ver con la reproducibilidad de los resultados de los experimentos de cristalización, pues no siempre son idénticos.
		
		\item Para clasificar los resultados de cristalización, se usó un esquema de clasificación modificado a partir del esquema de Bruno \emph{et al.} \cite{Bruno2018}. La clasifiación es la siguiente: 1 (cristales), 0.5 (cristales pequeños), 0 (gota clara), -1 (precipitado) y -2 (otros). Se tomaron tres fotos con una cámara USB acoplada a un microscopio óptico en los días quinto, noveno y décimosexto. Esto significa que se visualizaron y clasificaron \num{4320} fotos. Los resultados se definieron como positivos si se presentan cristales, en al menos uno de los dos experimentos, en los siguientes valores de pH 10, 9, 7.5, 5.0 y 4.5; de lo contrario, se consideran como resultados negativos. Para la tripsina, por lo menos al momento del día de la tercera foto, todos los resultados fueron negativos. Los resultados positivos para la lisozima se resumen a continuación.
		
		
		\begin{table}[h]
			\centering
			\begin{tabular}{@{}llll@{}}
				\toprule
				Proteína & Concentración & \% p/v NaCl & AP:AS \\ \midrule
				Lisozima & 15 & 3 & 1:1 \\
				Lisozima & 15 & 3 & 1:4 \\
				Lisozima & 15 & 6 & 1:4 \\
				Lisozima & 30 & 3 & 1:1 \\
				Lisozima & 30 & 6 & 1:1 \\
				Lisozima & 30 & 3 & 1:2 \\
				Lisozima & 30 & 6 & 1:2 \\
				Lisozima & 30 & 3 & 1:4 \\
				Lisozima & 30 & 6 & 1:4 \\ \bottomrule
			\end{tabular}
			\caption{Condiciones de cristalización exitosas para la lisozima. AP y AS significan aceite de parafina y silicona, respectivamente.}
			\label{tab:my-table}
		\end{table}
	\end{itemize}

	\item Con respecto a determinar los parámetros de la colecta de datos que produzcan niveles similares, o idénticos, de dosis de radiación absorbida, se escribió un programa en el lenguaje de programación \verb|bash| que permite obtener la dosis de radiación absorbida,  calculada por \verb|raddose| \cite{Bury2018}. Por simplicidad, se determinó un cristal hipotético de lisozima cúbico, con un volumen de \SI{1000000}{\cubic\angstrom}. Este presenta la misma composición atómica que la entrada \verb|1iee| del \verb|PDB|. También por simplicidad, el haz de rayos X se definió como colimado, con un corte de \num{100} x  \SI{100}{\micro\meter}, abarcando el área completa del cristal y un tiempo total de exposición de \SI{100}{\second}, con la obtención de un patrón de difracción por segundo. Dicho programa muestrea un flujo de fotones de 1.0 a \SI{100e11}{fotones\per\second} con un paso d \SI{0.1}{fotones\per\second} y una energía de 10.0 a \SI{17.2}{\kilo\electronvolt} con un paso de \SI{0.2}{\kilo\electronvolt}. Se obtuvo acceso al clúster \verb|groc| del grupo de genómica computacional del Instituto de Biotecnología de la UNAM\footnote{\url{https://biocomputo.ibt.unam.mx/}.}para correr dicho programa. Cabe aclarar que si bien los parámetros experimentales dados son idóneos, el fin de este programa es tener un estimado de la dosis por radiación absorbida en un sincrotrón de tercera generacion. Por otra parte, este primer programa sienta las bases para cualquier programa posterior en el que se necesite adaptar los parámetros experimentales a la realidad.
	 
	\item Con respecto a las colectas de datos, como primer acercamiento, se llevaron cinco\footnote{Crecidos a pH 10 (dos), 7.5 (uno) y 4.5 (dos).} cristales de lisozima al Laboratorio Nacional de Estructura de Macromoléculas (LANEM) del Instituto de Química (IQ) de la UNAM el 20 de mayo del presente año. En el LANEM-IQ, se tiene acceso a un generador de rayos X \verb|MicroMax-007 HF| de \verb|Rigaku|\footnote{\url{https://www.rigaku.com/products/sources/mm007}}. Hasta el momento, el único dato del que se tiene conocimiento es que la difracción de al menos uno de los cristales era mejor que \SI{1.6}{\angstrom}. 
	\end{enumerate}


%\pagelayout{wide} % No margins
%\addpart{Title of the Part}
%\pagelayout{margin} % Restore margins

\appendix % From here onwards, chapters are numbered with letters, as is the appendix convention

\chapter{Proteínas más representadas}
\labch{top50}

\begin{table}[h]
	\centering
	\begin{tabular}{@{}llllll@{}}
	\toprule
	Número & Código    & Cuenta & No. & Código       & Cuenta \\* \midrule

1      & \texttt{P11838} & 689    & 26     & \texttt{P23497}     & 106  \\
2      & \texttt{P00918} & 619    & 27     & \texttt{P00800}     & 103  \\
3      & \texttt{P00698} & 498    & 28     & \texttt{O26232}     & 102  \\
4      & \texttt{P00760} & 310    & 29     & \texttt{P22629}     & 97   \\
5      & \texttt{Q6PJP8} & 296    & 30     & \texttt{P00489}     & 96   \\
6      & \texttt{Q6B0I6} & 269    & 31     & \texttt{P03367}     & 96   \\
7      & \texttt{P02766} & 258    & 32     & \texttt{P68400}     & 95   \\
8      & \texttt{O95696} & 257    & 33     & \texttt{A0A073FPA6} & 84   \\
9      & \texttt{Q9UIF8} & 227    & 34     & \texttt{P46881}     & 79   \\
10     & \texttt{P00644} & 216    & 35     & \texttt{P00811}     & 78   \\
11     & \texttt{P00720} & 215    & 36     & \texttt{Q16539}     & 77   \\
12     & \texttt{P24941} & 203    & 37     & \texttt{P14174}     & 75   \\
13     & \texttt{P42212} & 182    & 38     & \texttt{P00431}     & 73   \\
14     & \texttt{P29476} & 178    & 39     & \texttt{P01116}     & 73   \\
15     & \texttt{P02185} & 172    & 40     & \texttt{P00183}     & 72   \\
16     & \texttt{O60885} & 170    & 41     & \texttt{P01112}     & 72   \\
17     & \texttt{P18031} & 152    & 42     & \texttt{Q00511}     & 70   \\
18     & \texttt{P61823} & 145    & 43     & \texttt{Q76353}     & 68   \\
19     & \texttt{P28720} & 144    & 44     & \texttt{P00282}     & 64   \\
20     & \texttt{P07900} & 142    & 45     & \texttt{P02883}     & 63   \\
21     & \texttt{P61626} & 139    & 46     & \texttt{P02945}     & 63   \\
22     & \texttt{P15121} & 125    & 47     & \texttt{P06873}     & 63   \\
23     & \texttt{P56817} & 124    & 48     & \texttt{P16113}     & 63   \\
24     & \texttt{P0DTD1} & 123    & 49     & \texttt{Q04609}     & 63   \\
25     & \texttt{P19491} & 116    & 50     & \texttt{P00772}     & 61   \\ \bottomrule
	\caption[Las 50 proteínas más representadas]{Las 50 proteínas más representadas en los datos, después de los cuatro primeros filtros. Eliminando estructuras que no contienen: código de acceso ni el valor de pH de la condición experimental. Además descarta aquellas estructuras con una resolución peor que \SI{2}{\angstrom} y donde el número de entidades es mayor o igual a dos.}
	\labtab{tab:top50}
	\end{tabular}
\end{table}


%\chapter{Análisis visual}
\labch{vis-anal}
\begin{figure}[h!]
	\centering
	\includegraphics[width=0.8\textwidth]{imgs/box_pH_by_gpo_P00698}
	\includegraphics[width=0.8\textwidth]{imgs/hist_pH_by_gpo_P00698}
	\caption[Análisis visual]{Ejemplo del análisis visual: histograma de \texttt{P00698}. Nótese, en el gráfico de caja, que esta proteína tiene un amplio intervalo de pH $\sim~5$, en particular cuando cristaliza en el grupo espacial \texttt{P 43 21 2}. En este caso, la mayor parte de las estructuras están cristalizadas a un pH cercano a \num{4.7}, de ahí la forma final del gráfico de caja. Su frecuencia de cristalización en este mismo intervalo, según el histograma, es arriba de cinco (denotado por la línea horizontal roja), por lo  menos para buena parte del intervalo.}
	\labfig{fig:vis-anal}
\end{figure}


%\pagelayout{wide} % No margins
%\addpart{Apéndices}
%\pagelayout{margin} % Restore margins


%----------------------------------------------------------------------------------------

\backmatter % Denotes the end of the main document content
\setchapterstyle{plain} % Output plain chapters from this point onwards

%----------------------------------------------------------------------------------------
%	BIBLIOGRAPHY
%----------------------------------------------------------------------------------------

% The bibliography needs to be compiled with biber using your LaTeX editor, or on the command line with 'biber main' from the template directory

\defbibnote{bibnote}{Listado de referencias por orden de aparición.\par\bigskip} % Prepend this text to the bibliography
\printbibliography[heading=bibintoc, title=Referencias, prenote=bibnote] % Add the bibliography heading to the ToC, set the title of the bibliography and output the bibliography note

%\defbibnote{bibnote}{Here are the references in citation order.\par\bigskip} % Prepend this text to the bibliography
%\printbibliography[heading=bibintoc, title=Bibliography, prenote=bibnote] % Add the bibliography heading to the ToC, set the title of the bibliography and output the bibliography note

%----------------------------------------------------------------------------------------
%	INDEX
%----------------------------------------------------------------------------------------

% The index needs to be compiled on the command line with 'makeindex main' from the template directory

\printindex % Output the index

\end{document}
