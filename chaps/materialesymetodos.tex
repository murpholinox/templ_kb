\chapter{Materiales y métodos}
\labch{mayme}

\section{Proteínas seleccionadas}
El \acrshort{pdb} contiene información acerca de cientos de miles de proteínas cristalizadas en diferentes condiciones de cristalización. Para hallar qué proteínas pueden cristalizar en un amplio intervalo de pH se realizó un análisis \emph{in silico}, como se explica brevemente\sidenote{Con detalle en \url{https://github.com/murpholinox/doctorado}.} a continuación. 

\subsection{Extracción de datos}
%\sidenote{Actualmente la API fue reemplazada por una nueva versión cuyo desarrollo no ha sido completado \url{https://tinyurl.com/y84kuvs7}.}
Se extrajo la información experimental directamente del cabezal de los archivos de las estructuras depositadas en el \acrshort{pdb}. Para extraer la información de los cabezales se usó el programa \Package{gemmi}\sidenote{Disponible en el siguiente enlace \url{https://github.com/project-gemmi/gemmi}.}. La información extraída es la siguiente: el contador de la entidad macromolecular, el tipo de entidad, el código de acceso de la base de datos de referencia\sidenote{La mayoría de las veces el código usado es aquél de la base de datos UniProt \url{https://www.uniprot.org/}.}, su descripción, el método experimental para crecer los cristales, el pH de la condición de cristalización, los detalles del experimento de la cristalización, la resolución final del modelo estructural, el grupo espacial en el que cristaliza la macromolécula y el identificador de objeto digital de la publicación científica correspondiente. A continuación se muestra un ejemplo de la información extraída:

\begin{kaobox}[frametitle=Ejemplo 1]
\begin{verbatim}
6LU7,1,polypeptide(L),P0DTD1,main protease,EVAPORATION,\
6,"2% polyethylene glycol (PEG) 6000, 3% DMSO, 1mM DTT,\
0.1M MES buffer (pH 6.0), protein concentration 5mg/ml,\
VAPOR DIFFUSION, HANGING DROP, temperature 293K",2.16,\
2.16,C 1 2 1,10.1038/s41586-020-2223-y,21728,210031
6LU7,2,polypeptide(L),P0DTD1,main protease,EVAPORATION,\
6,"2% polyethylene glycol (PEG) 6000, 3% DMSO, 1mM DTT,\
0.1M MES buffer (pH 6.0), protein concentration 5mg/ml,\
VAPOR DIFFUSION, HANGING DROP, temperature 293K",2.16,\
2.16,C 1 2 1,10.1038/s41586-020-2223-y,21728,210031
\end{verbatim}
\end{kaobox}

El resultado final es una tabla de datos, con \num{226523} observaciones, o filas, y \num{14} variables, o columnas. El número de observaciones es mayor que el número de estructuras depositadas en el \acrshort{pdb} y esto es debido a que cada entrada puede tener más de una macromolécula, como se puede notar en el \Environment{Ejemplo 1}. La integridad de los datos se puede verificar al graficar algunas de las variables extraídas.

\subsection{Limpieza de datos}
Se aplican los siguientes filtros a los datos extraídos para eliminar observaciones que:

\begin{enumerate}
	\item Carecen de un código de acceso.
	\item Tienen una resolución peor que \SI{2}{\angstrom}. 
	\item Carecen del valor de pH de la condición de cristalización.
	\item El número de entidades es mayor o igual a dos.
\end{enumerate}

La lógica de los filtros es la siguiente: (\emph{i} y \emph{iii}) Remover entradas que no tengan las anotaciones correspondientes\sidenote{Desafortunadamente, la información experimental no siempre se encuentra en los archivos del PDB.}, en este caso el código de acceso de la proteína y el pH de la condición de cristalización. Siendo la última la anotación más relevante para este proyecto y la primera, es la única manera de conocer casi inequívocamente la proteína representada en el archivo. (\emph{ii}) La resolución final del modelo estructural es un indicador \emph{grosso modo}, de la calidad del cristal obtenido. Este filtro garantiza que el listado de proteínas obtenidas sean fáciles de cristalizar. Además este filtro sirve para el análisis posterior, al comparar el daño por radiación específico entre proteínas/pHs, pues los resultados serán más confiables con una buena resolución. (\emph{iv}) Como se mencionó anteriormente, un archivo puede contener múltiples macromoléculas. Este filtro ayuda a descartar proteínas cristalizadas con otras. En general, las condiciones de cristalización para combinaciones diferentes de macromoléculas serán diferentes; por lo que no tiene caso tener varias observaciones de la misma proteína, si esta cristaliza con diferentes macromoléculas. 

En el apéndice se listan las 50 proteínas más representadas en el \acrshort{pdb}, cuando cumplen los primeros cuatro filtros (\reftab{tab:top50}). 

Posteriormente se aplican los siguientes dos filtros, que ayudan a eliminar las observaciones donde:

\begin{enumerate}
	\setcounter{enumi}{4}
	\item Su secuencia de aminoácidos sea diferente de la secuencia consenso para cada conjunto de proteínas.
	\item No presenten un intervalo de pH amplio en su cristalización.
\end{enumerate}

La lógica de estos dos filtros es la siguiente: (\emph{v}) elimina entradas donde el número de mutaciones, sustituciones o deleciones, sea mayor o igual a quince. Esto elimina, por ejemplo, aquellas proteínas cristalizadas con el péptido señal (en caso de tenerlo) y, por otra parte, mantiene entradas de aquellas proteínas cristalizadas con etiquetas de purificación por afinidad (siempre y cuando la etiqueta no sea mayor de quince péptidos).  Además elimina entradas donde la secuencia de la proteína, no corresponda a la secuencia consenso. En general, cada proteína tiene uno o varios identificadores UniProt. En el caso de las poliproteínas de algunos virus, el identificador de la poliproteína es el mismo para las subproteínas. A continuación se muestra un ejemplo:

\begin{kaobox}[frametitle=Ejemplo 2]
	\begin{verbatim}
	6W4H,1,polypeptide(L),P0DTD1,SARS-CoV-2 NSP16,...
	6W4H,2,polypeptide(L),P0DTD1,SARS-CoV-2 NSP16,...
	6W6Y,1,polypeptide(L),P0DTD1,SARS-CoV-2 NSP3,...
	6WQD,1,polypeptide(L),P0DTD1,SARS-CoV-2 NSP7,...
	6WQD,2,polypeptide(L),P0DTD1,SARS-CoV-2 NSP7,...
	7BUY,1,polypeptide(L),7BUY,SARS-CoV-2 virus Main protease,
	\end{verbatim}
\end{kaobox}


Es claro que el código, \texttt{P0DTD1}, es el mismo; sin embargo, las observaciones corresponden a diferentes proteínas.

El alineamiento múltiple se realizó con el programa \Package{mafft} \sidecite{Katoh2013}. La secuencia consenso se obtuvo con el programa \Package{cons} de la suite de programas \Package{emboss} \sidecite{Madeira2019}. Para la determinación del número de mutaciones se aplicó la función \Package{adist} \sidenote{\url{https://www.rdocumentation.org/packages/utils/versions/3.6.2/topics/adist}} en el lenguaje de programación \Package{R}. 

Y, (\emph{vi}), es la condición experimental que nos interesa en este proyecto, proteínas que cristalicen en un amplio intervalo de pH. Este filtro se aplicó por partes. Primero se realizó un gráfico de caja para cada una de las 50 proteínas más representadas en los datos. Este tipo de gráfica da una representación visual de la distribución de la variable en cuestión. Si la distribución no cubre al menos tres unidades de pH, entonces se descarta dicha proteína, en caso contrario se mantiene. De las proteínas restantes, 25, se realiza un histograma de frecuencias para determinar de manera cuantitativa la frecuencia con la que cada proteína ha sido cristalizada en valores de pH diferentes. Si la frecuencia no es mayor a cinco para la mayoría de las barras en cada histograma, la proteína se descarta, si no se mantiene. Esto resulta en un listado de 14 proteínas, donde las diez primeras entradas se presentan en la correspondiente sección de resultados (\reftab{tab:list}). 

\section{Cristalización de proteínas}
\subsection{Lisozima}
La lisozima de clara de huevo de gallina (\emph{Gallus gallus}), se obtuvo de manera comercial de \Package{Sigma-Aldrich}, con el número de catálogo \Package{L6876} de manera liofilizada. Según la hoja de datos tiene una pureza mayor o igual al \SI{90}{\percent} y por lo tanto no se realizó ningún paso de purificación subsecuente.

\subsection{Tripsina}
La tripsina pancreática de bovino (\emph{Bos taurus}), también se obtuvo de manera comercial de \Package{Sigma-Aldrich}, con el número de catálogo \Package{T4799} de manera liofilizada.\todo{Cuál es la pureza?} Tampoco se realizó ningún paso de purificación subsecuente.

\subsection{Amortiguadores de capacidad extendida}
De los sistemas de amortiguadores de capacidad extendida desarrollados por Newman \cite{Newman2004}, se utilizó la primera condición de cristalización (denominada de ahora en adelante, S1): ácido succínico, glicina y sodio dihidrógeno fosfato monohidratado a una concentración de \SI{250}{\milli\Molar}. \todo{Y posiblemente más!}

\subsection{Experimentos de cristalización}
Los experimentos de cristalización se realizaron con la técnica modificada de \emph{microbatch}, propuesta por D'Arcy \emph{et al.} \cite{DArcy1996}. 

En estos experimentos, se varió la concentración de la proteína (15 o \SI{30}{\milli\gram\per\milli\liter}); la proporción entre aceites de parafina y silicona (1:0, 1:0.5, 1:1, 1:2 y 1:4); el pH (10, 9.5, 9.0, $\ldots$, 4.5) y finalmente, también se cambió el porcentaje de cloruro de sodio (0, 3 y \SI{6}{\percent} p/v) en la condición de cristalización. Esto debido a que, en ciertas condiciones, se ha observado una dependencia del tamaño de los cristales de lisozima obtenidos con respecto a la concentración de cloruro de sodio \cite{Svanidze2005}. Se usaron placas de cristalización de Terasaki Greiner de \num{72} pozos. Esto permite tres corridas de pH, con tres porcentajes de cloruro de sodio por duplicado. Cabe destacar que en cada placa la proporción de aceites es diferente. 

\subsection{Clasificación de resultados de cristalizacion}
Para clasificar los resultados de cristalización, se usó un esquema de clasificación modificado a partir del esquema de Bruno \emph{et al.} \cite{Bruno2018}. La clasifiación es la siguiente: 1 (cristales), 0.5 (cristales pequeños), 0 (gota clara), -1 (precipitado) y -2 (otros). Se tomaron cuatro fotos con una cámara USB acoplada a un microscopio óptico en los días quinto, noveno, décimosexto y trigésimo octavo.

\section{Colecta y análisis de datos}
Los cristales obtenidos serán difractados en un sincrotrón. Para cada cristal macromolecular se tendrán que realizar series de colectas de datos de manera repetitiva. Gracias a la presencia de modelos estructurales en el \acrshort{pdb}, se puede estimar la dosis absorbida por la proteína antes de realizar el experimento de difracción. Esto se realiza gracias al programa \Package{raddose} \sidecite{Zeldin2013}. Se resolveran las estructuras por reemplazo molecular y se crearán mapas de diferencia de densidad electrónica para analizar la diferencia en daño por radiación específico a distintos pHs al mismo nivel de dosis de radiación absorbida. %Cabe señalar que la mayor parte del análisis posterior a la colecta de datos será \emph{in silico} y gracias al trabajo anterior se dispone de un flujo de trabajo semiautomático lo que facilitará en gran medida el presente proyecto.
\todo{Aquí obviamente falta poner los parámetros experimentales de las colectas de datos. }
