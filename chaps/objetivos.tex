\chapter{Objetivo}
\labch{objet}

El objetivo de este proyecto es determinar el efecto del pH en la condición de cristalización de ciertas proteínas sobre el daño por radiación. 
\section{Objetivos particulares}
\labsec{objpar}
Para lograr el objetivo principal de este proyecto, se plantean los siguientes objetivos particulares:

Obtener mapas de diferencia de densidad electrónica entre puntos fijos de dosis de radiación absorbida.

\begin{enumerate}
	\item Realizar un análisis \emph{in silico} para determinar qué proteínas son capaces de cristalizar en un intervalo de pH amplio.
	\item Cristalizar las proteínas seleccionadas a diferentes niveles de pH usando cualquiera de los sistemas de amortiguamiento que muestrean un intervalo de pH amplio
	\item Determinar los parámetros de la colecta de datos que produzcan niveles similares o idénticos de dosis de radiación absorbida en las diferentes proteínas cristalizadas.
	\item Realizar colectas de datos continuas y seriales en un sincrotrón.
	\item Procesar los patrones de difracción para obtener un modelo inicial de las proteínas.
	\item Mapear la diferencia de densidad electrónica entre colectas de datos al modelo inicial de cada proteína y realizar un análisis comparativo, en particular sobre los residuos de aminoácido que son más susceptibles al daño por radiación, para determinar las diferencias en daño por radiación a diferentes niveles de pH.
\end{enumerate}	