\chapter{Resultados}

\section{Proteínas seleccionadas} 

Del análisis \emph{in silico} para determinar qué proteínas son capaces de cristalizar en un intervalo de pH amplio, se obtuvieron diez proteínas.

\begin{table}[h]
	\centering
	\begin{tabular}{@{}llll@{}}
		\toprule
		Número & Código & Intervalo       & Nombre                              \\ \midrule
		1      & P00918 & 6               & Anhidrasa carbónica                 \\
		2      & P00698 & 5.5             & Lisozima                            \\
		3      & P00760 & 6               & Tripsina                            \\
		4      & P02766 & 5               & Transtiretina                       \\
		5      & P42212 & 6               & Proteína verde fluorescente         \\
		6      & O60885 & 3.5             & Proteína bromodominio 4             \\
		7      & P19491 & 4.5             & Receptor de glutamato 2             \\
		8      & O26232 & 4               & Orotidina-5’-fosfato descarboxilasa \\
		9      & P00772 & 4.5             & Elastasa                            \\
		10     & P00644 & 3               & Termonucleasa                       \\
		\bottomrule
	\end{tabular}
	\caption[Proteínas que cristalizan en un intervalo amplio de pH]{Proteínas que cristalizan en un intervalo amplio de pH.}
	\labtab{tab:list}
\end{table}


\section{Cristalización}
Con la variación de parámetros experimentales para la cristalización, se obtuvieron \num{360} experimentos (\num{720}, tomando en cuenta los duplicados), por proteína. %Esto último, tiene que ver con la reproducibilidad de los resultados de los experimentos de cristalización, pues no siempre son idénticos.

\subsection{Lisozima}
Esto significa que se visualizaron y clasificaron \num{5760} fotos. Los resultados se definieron como positivos si se presentan cristales, en al menos uno de los dos experimentos, en los siguientes valores de pH 10, 9, 7.5, 5.0 y 4.5; de lo contrario, se consideran como resultados negativos. 
Los resultados positivos para la lisozima se resumen a continuación.


\begin{table}[h]
	\centering
	\begin{tabular}{@{}llll@{}}
		\toprule
		Proteína & Concentración & \% p/v NaCl & AP:AS \\ \midrule
		Lisozima & 15 & 3 & 1:1 \\
		Lisozima & 15 & 3 & 1:4 \\
		Lisozima & 15 & 6 & 1:4 \\
		Lisozima & 30 & 3 & 1:1 \\
		Lisozima & 30 & 6 & 1:1 \\
		Lisozima & 30 & 3 & 1:2 \\
		Lisozima & 30 & 6 & 1:2 \\
		Lisozima & 30 & 3 & 1:4 \\
		Lisozima & 30 & 6 & 1:4 \\ \bottomrule
	\end{tabular}
	\caption{Condiciones de cristalización exitosas para la lisozima. AP y AS significan aceite de parafina y silicona, respectivamente.}
	\label{tab:my-table}
\end{table}


\subsection{Tripsina}
Para la tripsina, por lo menos al momento del día de la tercera foto, todos los resultados fueron negativos. 

\section{Dosis de radiación}
Se escribió un programa en el lenguaje de programación \verb|bash| que permite obtener la dosis de radiación absorbida,  calculada por \verb|raddose| \cite{Bury2018}. Por simplicidad, se determinó un cristal hipotético de lisozima cúbico, con un volumen de \SI{1000000}{\cubic\angstrom}. Este presenta la misma composición atómica que la entrada \verb|1iee| del \verb|PDB|. También por simplicidad, el haz de rayos X se definió como colimado, con un corte de \num{100} x  \SI{100}{\micro\meter}, abarcando el área completa del cristal y un tiempo total de exposición de \SI{100}{\second}, con la obtención de un patrón de difracción por segundo. Dicho programa muestrea un flujo de fotones de 1.0 a \SI{100e11}{fotones\per\second} con un paso d \SI{0.1}{fotones\per\second} y una energía de 10.0 a \SI{17.2}{\kilo\electronvolt} con un paso de \SI{0.2}{\kilo\electronvolt}. Se obtuvo acceso al clúster \verb|groc| del grupo de genómica computacional del Instituto de Biotecnología de la UNAM\footnote{\url{https://biocomputo.ibt.unam.mx/}.}para correr dicho programa. Cabe aclarar que si bien los parámetros experimentales dados son idóneos, el fin de este programa es tener un estimado de la dosis por radiación absorbida en un sincrotrón de tercera generacion. Por otra parte, este primer programa sienta las bases para cualquier programa posterior en el que se necesite adaptar los parámetros experimentales a la realidad.

\section{Colectas de datos} 
%Con respecto a las colectas de datos, como primer acercamiento, se llevaron cinco\footnote{Crecidos a pH 10 (dos), 7.5 (uno) y 4.5 (dos).} cristales de lisozima al Laboratorio Nacional de Estructura de Macromoléculas (LANEM) del Instituto de Química (IQ) de la UNAM el 20 de mayo del presente año. En el LANEM-IQ, se tiene acceso a un generador de rayos X \verb|MicroMax-007 HF| de \verb|Rigaku|\footnote{\url{https://www.rigaku.com/products/sources/mm007}}. Hasta el momento, el único dato del que se tiene conocimiento es que la difracción de al menos uno de los cristales era mejor que \SI{1.6}{\angstrom}. 
